\chapter{Kinematics of particles}
%~~~~~~~~~~~~~~~~~~~~~~~~~~~~~~~~~~~~~~~~~~~~~~~~~~~~~~~~~~
\section{Inertia}

\begin{flushright}
\begin{minipage}[h]{10cm}
% \raggedright
{\small
Inertia is that tendency of anything to maintain a constancy of its state. \textit{In other words}, is that tendency of anything to resist a change in the constancy of its state. \textit{Level 1} \textbf{The inertia of a body is that tendency of the body to maintain constancy of its state of uniform motion}. \textit{Level 2}  \textbf{The inertia of a body is that tendency of the body to maintain its current state of motion, that is, to either maintain its state of rest or of uniform motion \footnote{Here, uniform motion means motion with constant velocity, hence implying zero-acceleration in the direction of motion.}}}
\end{minipage}

\end{flushright}

Consider rectilinear motion of a body momentarily at rest. When a force acts on it, it tends to move, but however, the body does offer a resistance  to the applied force. The amount of resistance offered by the body to change its state of rest is decided by the mass of this body. The corresponding acceleration which the body acquires as a consequence is therefore decided by the mass of the body itself and nevertheless by the amount of the applied force as well. Here, had the mass of the object would have been any lesser, than for the same applied force, it could have acquired greater value of acceleration in the direction of the applied force. ON the other hand, had the mass of the body would have been an y greater then, for the same amount of applied force, the body would have experienced a lesser amount of acceleration in the direction of the applied force.

When two bodies A and B of equal masses $m_A$ and $m_B$ ($m_A = m_B$), then for the same force, there will be same values of accelerations. 

When two bodies A and B of equal masses $m_A$ and $m_B$ ($m_A = m_B$), are acted by the forces $F_A$ and $F_B$ ($F_A > F_B$), then the body A acquires greater value of acceleration than B. IN other words, to accelerate a body having mass to greater and greater values, higher and higher forces are needed. This dependency between force need and acceleration to be acquired is found top be linear and the constant of proportionality is nothing but the mass of the object itself. 


%\begin{figure}
%\includegraphics[h!]{image.jpeg}
%\end{figure}

IMAGE


Hence, the mass of the object is the slope of the above linear relationship between $\mathbf{F}$ and $\mathbf{a}$.

But why bring in acceleration here any way? We saw in the definition of inertia of a body that it not only represents the resistance offered by the body against the forces which are trying to change its ststae of rest, but also represents the resistance offered against the forces which are trying to change its state of motion under uniform velocity. 
 
When a body in rectilinear motion is getting displaced by equal amounts in any arbitrarily chosen but equal intervals of time, and if any unbalanced force now starts acting on this body, then after this instant of time, for any two arbitrarily chosen equal intervals of time, the body undergoes in-equal displacements and hence the velocity of such body is different at different instances of time, both within the time intervals for which the applied force exists on the body. Hence, the displacements in the time interval tAB is sAB and that in tBC is sBC, then the instantaneous values of velocity are vAB = dsAB div dtAB, where dtAB in infinitesimal value of tAB and dsAB is the corresponding infinitesimal value of sAB; and vBC = dsBC div dtBC, where dtBC in infinitesimal value of tBC and dsBC is the corresponding infinitesimal value of sBC. Since clearly, sAB and sBC are not same here, vAB NE vBC for tAB = tBC. Hence there exists a time rate of change of velocity which between A and C is aAB = dvAC/dtAC, where once again dvAC is the change of velocities between AB and BC intervals and dtAC is the infinitesimal time interval taken for displacement between A and C.

\section{Mass moments of inertia: $I_{xx}, I_{yy}, I_{zz}$}
\subsection{Conceptual development}
\subsection{Mathematical development}
\subsection{Physical interpretations}

\section{Products of mass moments of inertia: $I_{xy}, I_{yz}, I_{zx}$}
\subsection{Conceptual development}
\subsection{Mathematical development}
\subsection{Physical interpretations}

\section{Inertia tensor}
\subsection{Conceptual development}
\subsection{Mathematical development}
\subsection{Physical interpretations}

\section{Principal moments and axes of inertia}
\subsection{Conceptual development}
\subsection{Mathematical development}
\subsection{Physical interpretations}

\section{Rectilinear motion and Plane curvilinear motion}