\documentclass[]{book}

\usepackage[utf8]{inputenc}
\usepackage{graphicx}

\usepackage{xcolor}
\usepackage{courier}
\usepackage{fancyvrb}
\usepackage{tcolorbox}

\usepackage{tikz}
\usetikzlibrary{mindmap,trees}

%\graphicspath{{./figures/}}
\usepackage[]{natbib}
%\usepackage{chapterbib}
\bibliographystyle%
  {wileynum}%for numerical citation and numerically listed entries in the bibliography
  %{wileyauy}%for author--year citation and alphabetical order in the bibliography
\usepackage{wileySTM}
\usepackage[\ifnum\pdfoutput=1breaklinks\fi]{hyperref}
\author{Sunil Anandatheertha}\title{A Textbook and Treatise on Engineering and Analytical  Mechanics}

%\includeonly{}

\begin{document}
\frontmatter
\include{foreword}
\tableofcontents
\include{preface}
\mainmatter

% % % % % % % % % % % % % % % % % % % % % % %
%\textcolor{blue}{}
% % % % % % % % % % % % % % % % % % % % % % %
{\flushleft\textbf{Foreword}}
\vspace{1cm}

\newpage

{\flushleft\textbf{Preface}}
\vspace{1cm}

This book is intended to bridge the gap between the academic treatment of basic engineering dynamics and advanced engineering dynamics, for both students, instructors, engineers and the like. The treatment provided mostly follows and builds on the treatment of the subject in the book "Engineering Mechanics Dynamics SI Version by J. L. Meriam and L. G. Kraige. 7$^{th}$" edition.

The physical world that we see, experience and take part in, is seamlessly governed by many underlying laws which goes un-noticed to the untrained `eye'. It is one of the motive of this book to enable the reader to train their eyes which would empower them to understand many aspects of the happenings in the physical world in a reasoned, quantitative, qualitative and predictable manner. The text is designed to expose the reader to a number of methods, basic and advanced, qualitative and quantitative, in which a problem can be looked at, unde0rstood and solved. Approaches provided in this work range from intuitive to force-mass-acceleration approach to impulse and momentum approach to energy methods. A problem which can be solved very easily using elementary approaches is also solved using powerful approaches such as Lagrangian's and Hamilton's approach so that these difficult concepts can be easily understood and applied. 

\textbf{One of the most important pedagogical approaches} in this book is the generalizations of a problem. A seemingly simple problem of engineering mechanics dynamics is taken to hand, solved, generalized to a next level, solved, again generalized to a next level and solved, and this in this way, many levels of generalizations of a problem are provided. In the experience of the author, this provides the reader with a hierarchical knowledge of underlying concepts, methods, their applicability, limitations and powers.

Computer methods of solving a problem in engineering mechanics dynamics which can be done using pen and paper and numerical approach is also provided to a great extent in order to aid \textbf{A} detailed analysis and solution to the problem at hand, \textbf{B} learning, understanding and application of numerous computer tools and numerical techniques available today, and \textbf{C} advanced computer visualizations of the solution obtained.

\textbf{Note on reading `analysis' sections}: Analysis sections are written every now and then, with an intension to provide a different and/or deeper and/or different perspective and understanding of a concept or definition or observation or behavior or even a mathematical result/expression/development. The content, grammar and construct of analysis sections is written to render themselves slightly in-accessible to most beginner readers/students in the first reading, but has been designed and written so as convey a deeper understanding once they are visited after a few problems are  solved, understood and mastered. Nevertheless, it is needed to gain a deeper insight and understanding into the subject at hand.
%~~~~~~~~~~~~~~~~~~~~~~~~~~~~~~~~~~~~~~~~~~~~~~~~~~~~~~~~~~~~~~~~~~~~~~
%~~~~~~~~~~~~~~~~~~~~~~~~~~~~~~~~~~~~~~~~~~~~~~~~~~~~~~~~~~~~~~~~~~~~~~
%~~~~~~~~~~~~~~~~~~~~~~~~~~~~~~~~~~~~~~~~~~~~~~~~~~~~~~~~~~~~~~~~~~~~~~
%~~~~~~~~~~~~~~~~~~~~~~~~~~~~~~~~~~~~~~~~~~~~~~~~~~~~~~~~~~~~~~~~~~~~~~
%~~~~~~~~~~~~~~~~~~~~~~~~~~~~~~~~~~~~~~~~~~~~~~~~~~~~~~~~~~~~~~~~~~~~~~
%~~~~~~~~~~~~~~~~~~~~~~~~~~~~~~~~~~~~~~~~~~~~~~~~~~~~~~~~~~~~~~~~~~~~~~
%~~~~~~~~~~~~~~~~~~~~~~~~~~~~~~~~~~~~~~~~~~~~~~~~~~~~~~~~~~~~~~~~~~~~~~
%~~~~~~~~~~~~~~~~~~~~~~~~~~~~~~~~~~~~~~~~~~~~~~~~~~~~~~~~~~~~~~~~~~~~~~
%~~~~~~~~~~~~~~~~~~~~~~~~~~~~~~~~~~~~~~~~~~~~~~~~~~~~~~~~~~~~~~~~~~~~~~
%~~~~~~~~~~~~~~~~~~~~~~~~~~~~~~~~~~~~~~~~~~~~~~~~~~~~~~~~~~~~~~~~~~~~~~
%~~~~~~~~~~~~~~~~~~~~~~~~~~~~~~~~~~~~~~~~~~~~~~~~~~~~~~~~~~~~~~~~~~~~~~
%~~~~~~~~~~~~~~~~~~~~~~~~~~~~~~~~~~~~~~~~~~~~~~~~~~~~~~~~~~~~~~~~~~~~~~
\part{Review of statics and dynamics} %Part1_C**.tex
%~~~~~~~~~~~~~~~~~~~~~~~~~~~~~~~~~~~~~~~~~~~~~~~~~~~~~~~~~~~~~~~~~~~~~~
%~~~~~~~~~~~~~~~~~~~~~~~~~~~~~~~~~~~~~~~~~~~~~~~~~~~~~~~~~~~~~~~~~~~~~~
%~~~~~~~~~~~~~~~~~~~~~~~~~~~~~~~~~~~~~~~~~~~~~~~~~~~~~~~~~~~~~~~~~~~~~~
%~~~~~~~~~~~~~~~~~~~~~~~~~~~~~~~~~~~~~~~~~~~~~~~~~~~~~~~~~~~~~~~~~~~~~~
%~~~~~~~~~~~~~~~~~~~~~~~~~~~~~~~~~~~~~~~~~~~~~~~~~~~~~~~~~~~~~~~~~~~~~~
%~~~~~~~~~~~~~~~~~~~~~~~~~~~~~~~~~~~~~~~~~~~~~~~~~~~~~~~~~~~~~~~~~~~~~~
%~~~~~~~~~~~~~~~~~~~~~~~~~~~~~~~~~~~~~~~~~~~~~~~~~~~~~~~~~~~~~~~~~~~~~~
%~~~~~~~~~~~~~~~~~~~~~~~~~~~~~~~~~~~~~~~~~~~~~~~~~~~~~~~~~~~~~~~~~~~~~~
%~~~~~~~~~~~~~~~~~~~~~~~~~~~~~~~~~~~~~~~~~~~~~~~~~~~~~~~~~~~~~~~~~~~~~~
%~~~~~~~~~~~~~~~~~~~~~~~~~~~~~~~~~~~~~~~~~~~~~~~~~~~~~~~~~~~~~~~~~~~~~~
%~~~~~~~~~~~~~~~~~~~~~~~~~~~~~~~~~~~~~~~~~~~~~~~~~~~~~~~~~~~~~~~~~~~~~~
%~~~~~~~~~~~~~~~~~~~~~~~~~~~~~~~~~~~~~~~~~~~~~~~~~~~~~~~~~~~~~~~~~~~~~~
\part{Engineering Mechanics-Dynamics: Kinematics of particles} %Part2_C**.tex
%\begin{tikzpicture}
%  \path[mindmap,concept color=black,text=white]
%    node[concept, scale = 0.65] {Kinematics of particles}
%    [clockwise from=0]
%    child[concept color=green!50!black, scale = 0.8] {
%      node[concept] {practical}
%      [clockwise from=90]
%      child { node[concept, scale = 0.6] {algorithms} }
%      child { node[concept, scale = 0.6] {data structures} }
%      child { node[concept, scale = 0.6] {pro\-gramming languages} }
%      child { node[concept, scale = 0.6] {software engineer\-ing} }
%    }  
%    child[concept color=blue] {
%      node[concept] {applied}
%      [clockwise from=-30]
%      child { node[concept] {databases} }
%      child { node[concept] {WWW} }
%    }
%    child[concept color=red] { node[concept] {technical} }
%    child[concept color=orange] { node[concept] {theoretical} };
%\end{tikzpicture}
% Introduction to kinematics of particles
	\chapter{Introduction}

\subsection{Frame of reference}
A \textit{frame of reference} in $\mathbf{R}^3$ space is a reference system having three mutually independent co-ordinates which can uniquely identify any point in that space. A \textit{fixed frame of reference} $\mathbf{R_A}$ is one which is non-moving and fixed in space, does not translate or rotate, and also the one with respect to which or with reference to which, all other measurements of motion can be made. If there exists another frame of reference $\mathbf{R_B}$ who's orientation and/or motion (translation and/or rotation) is known with respect to the $R_A$, and if the motion of a point moving in space can be easily observed from $R_B$, then the observations made in $R_B$ has to be transformed using transformations defined by the orientation of $R_B$ with respect to $R_A$, in order for the observations from $R_B$ to be interpretable as observations from the fixed frame of reference $R_A$. When a frame of reference such as $R_B$ moves with constant zero (stationary) /non-zero velocity with respect to $R_A$, then $R_B$ is an inertial frame of reference. Thus, an \textit{inertial frame of reference} is one which has no relative motion with the fixed frame of reference or is one which has constant velocity measured from the fixed frame of reference. Other frames of references such as \textit{rotating frame of reference} are used in the kinematic and kinetic analysis, and they will be introduced, explained and used when necessary. This topic is discussed and explained in more details later in the text.
	
\subsection{Scalars and vectors}
A \textit{scalar} is a mathematical representation of any `quantity' (Physical/Non-Physical) having significance only in its magnitude, but not in direction. This does not mean there could be no direction. It only means no significance of direction is desired/attached. However, a \textit{vector} is a mathematical representation of a physical quantity which can only be completely specified when we have full knowledge of its magnitude (or mathematical behavior of its magnitude) and also its direction/sense (or mathematical behavior of its direction). Many entities associated with motion fall into the category of vectors. In contrast to a scalar, a specific co-ordinate system is needed to completely, mathematically and quantitatively specify a vector.
	
\subsection{Distance and displacement}
Consider an arbitrary point in space at a position P1(x1, y1, z1) making is way to another position P2(x2, y2, z2) in a certain path. To do so, this particle traveled a distance 's' along the path that it took to reach P2 from P1. Upon starting from P1 and reaching P2, the point is displaced along a direction defined by the spatial co-ordinates of P1 and P2 (and also on the position of the reference used), by a length which again defined by spatial co-ordinates of P1 and P2. This length is the Euclidean straight line distance between P1 and P2 taken in rectangular co-ordinate system and is defined as $|\mathbf{s}| = \sqrt[2]{(x_2-x_1)^2+(y_2-y_1)^2+(z_2-z_1)^2}$. If the reference used for observing the motion of point from P1 to P2 is the fixed frame of reference, then the 

\subsection{Speed and velocity} When particle moves in 

\subsection{Velocity}
Velocity  of an arbitrary point of interest is a measure of time rate of change of displacement of that point. 

\textit{Analysis - 1a}: For a particle, the knowledge of it's magnitude of velocity implies a  knowledge of the relation between the actual displacement underwent and the actual time taken. It means to have knowledge of \textit{how much the particle actually got displaced} in \textit{exactly that much amount of time}. We generalize this argument in analysis - 1b below.

\textit{Analysis - 1b}: For a particle, the knowledge of it's magnitude of velocity implies a  knowledge of the relation between any arbitrary displacement underwent and the corresponding time taken, provided that all measurements are done within the time interval throughout which the velicity remains constant. It means to have knowledge of \textit{how much the particle would get displaced} in \textit{exactly so much amount of time}.

If the measure of displacement is rectilinear, the velocity is rectilinear velocity $\vec{\mathbf{v}}$. If the measure of displacement is curvilinear, the velocity is tangential velocity $\vec{\mathbf{v}}_t$. However, if the measure of displacement is angular, the velocity is angular velocity $\vec{\mathbf{\omega}}$. The SI units of $\vec{\mathbf{v}}$ and $\vec{\mathbf{v}}_t$ is $m/s$ and that of $\vec{\mathbf{\omega}}$ is $rad/s$. Since, displacement is a vector, it follows from the rules of vector calculus that, velocity is also a vector.

\begin{flushright}
\begin{minipage}[h!]{10.5cm}
\small{
\textbf{Physical interpretation - 1}\\ \textit{The direction of velocity vector is the same as that of the direction of displacement vector}. A particle has started to execute rectilinear motion from point 0. It reaches point 1 and then to point 2. Let, time taken to reach points 1 and 2 be started to be recorded from point 0, and say they are $t_{01}$ and $t_{02}$; \textbf{O} be the origin of an inertial frame which is used to observe the motion of the particle as it moves from point 0 to point 2 (All our measurements and hence, the following statements are made with respect to \textbf{O}) and $\mathbf{r}_{1/O}$ and $\mathbf{r}_{2/O}$ be the position vectors of instantaneous positions of the particle at points 1 and 2. Clearly the difference $\Delta t_{02-01} = \Delta t_{2-1} = t_{02}-t_{01}$, ($+^{ve}$ scalar) and is the time elapsed as the particle moved from points 1 to 2. However, the difference $\Delta r_{(2/O)/(1/O)} = \Delta r_{2/1} = r_{2/O}-r_{1/O}$ is vectorial, who's magnitude is just the difference in magnitudes of the two vectors, but however, the direction of $\Delta r_{2/1}$ is that of the vector joining the 2nd position (point 2) from the 1st position (point 1). This direction is that of the direction of motion between points 1 and 2. Hence, the direction of velocity vector of the particle executing motion from point 1 to 2 is the same as that of the displacement vector between the two points. When the spatial and time intervals between the above two states, states 1 and 2, is shrink to infinitesimal values (by bringing point 2 infinitesimally close to point 1) to be represented by an infinitesimal position vector, $d\mathbf{r}_{2/1}$ and infinitesimal time interval $dt_{2-1}$, we get the instantaneous velocity vector at point 1, which can be represented by $\mathbf{v}_{1}$, given by the ratio of $d\mathbf{r}_{2/1}$ and $dt_{2-1}$. Hence,\[\mathbf{v}_{1}=\frac{d\mathbf{r}_{2/1}}{dt_{2-1}} \]
}
\end{minipage}
\end{flushright}

\textbf{Analysis - 1}:

\subsection{Constant velocity}

\subsubsection{Constant rectilinear velocity or Zero-rectilinear Acceleration}
Consider a particle executing rectilinear motion. When the velocity vector $\textbf{v}$ is constant, its time derivative, that is $\frac{d\mathbf{v}}{dt}$, equal to acceleration vector $\mathbf{a}$, is zero.

Please go through the `\textbf{Note on reading `analysis' sections}' provided in Preface section of this book before you go any further in reading the following analysis points.


\begin{flushright}
\begin{minipage}[h]{10.5cm}
\small{
\textbf{Analysis - 1}: If a particle moving from point A to point B, from time instance $t_A$ to $t_B$, that is, in a time interval of $t_{AB}=t_B-t_A$, displaces itself with respect to position A by a Euclidean straight line distance of $|\mathbf{s}_{AB}|$, and if its displacement from position A to any arbitrarily chosen intermediate point C on the path its takes from A to B is $|\mathbf{s}_{AC}|$ and the time taken for this displacement is $t_{AC}$, and if the ratio of $|\mathbf{s}_{AB|}$ to $t_{AB}$ equals the ratio of $|\mathbf{s}_{AC}|$ to $t_{AB}$, then the displacement underwent by such a particle between any two arbitrary positions (say P and Q) on the path under consideration, follows a linear relationship with the time taken by the particle to be displaced from P to Q and the rate at which this ratio changes with time is essentially zero since this ratio is a constant, and in such cases we say that the particle has a constant velocity and consequently has no or zero-acceleration. The linear relationship mentioned here can be expressed as: $|\mathbf{s}_{PQ}|=\left(\frac{|\mathbf{s}_{AB}|}{t_{AB}}\right)t_{PQ}=\left(\frac{|\mathbf{s}_{AC}|}{t_{AC}}\right)t_{PQ}$. The term in brackets is the slope of the linear relationship between $|\mathbf{s}_{PQ}|$ and $t_{PQ}$.

\textbf{Analysis - 2}: Given an arbitrary time interval $t_{AB}$ and another arbitrary time interval $t_{PQ}$, both belonging to the time elapsed between positions A and B on the path 1-2 and P and Q on the same path 1-2, then under constant velocity motion between 1 and 2, for $t_{AB}=t_{PQ}$, the displacement of the particle from A to B will equal that of P to Q.

\textbf{Analysis - 3}: If $|\mathbf{s}_{AB}|$ and $|\mathbf{s}_{PQ}|$ be any two arbitrarily chosen equal displacement magnitudes (between points A and B, and P and Q respectively) underwent by a particle moving along a certain space, then, the time taken by the particle to be displaced from A to B and that between P and Q, should be equal.
}
\end{minipage}
\end{flushright}






\subsection{Acceleration}


\subsection{Constant acceleration}: 
\subsection{Variable acceleration: a = a(t), a = a(v) and a = a(s)}
\subsection{Practical cases of constant acceleration}
\subsection{Practical cases of variable acceleration}
\subsection{Working with variable acceleration problems}
%~~~~~~~~~~~~~~~~~~~~~~~~~~~~~~~~~~~~~~~~~~~~~~~~~~~~~~~~~~
%~~~~~~~~~~~~~~~~~~~~~~~~~~~~~~~~~~~~~~~~~~~~~~~~~~~~~~~~~~
\section{Co-ordinate systems}
\section{Rectangular coordinates}
\section{Normal and tangential co-ordinates}
\section{Polar co-ordinates}
\section{Space curvilinear motion}
\section{Relative motion with translating axis}
\section{Constrained motion of connected bodies}
% Introduction to kinematics of particles
	\chapter{Kinematics of particles}
%~~~~~~~~~~~~~~~~~~~~~~~~~~~~~~~~~~~~~~~~~~~~~~~~~~~~~~~~~~


\section{Rectilinear motion and Plane curvilinear motion}
%~~~~~~~~~~~~~~~~~~~~~~~~~~~~~~~~~~~~~~~~~~~~~~~~~~~~~~~~~~~~~~~~~~~~~~
%~~~~~~~~~~~~~~~~~~~~~~~~~~~~~~~~~~~~~~~~~~~~~~~~~~~~~~~~~~~~~~~~~~~~~~
%~~~~~~~~~~~~~~~~~~~~~~~~~~~~~~~~~~~~~~~~~~~~~~~~~~~~~~~~~~~~~~~~~~~~~~
%~~~~~~~~~~~~~~~~~~~~~~~~~~~~~~~~~~~~~~~~~~~~~~~~~~~~~~~~~~~~~~~~~~~~~~
%~~~~~~~~~~~~~~~~~~~~~~~~~~~~~~~~~~~~~~~~~~~~~~~~~~~~~~~~~~~~~~~~~~~~~~
%~~~~~~~~~~~~~~~~~~~~~~~~~~~~~~~~~~~~~~~~~~~~~~~~~~~~~~~~~~~~~~~~~~~~~~
%~~~~~~~~~~~~~~~~~~~~~~~~~~~~~~~~~~~~~~~~~~~~~~~~~~~~~~~~~~~~~~~~~~~~~~
%~~~~~~~~~~~~~~~~~~~~~~~~~~~~~~~~~~~~~~~~~~~~~~~~~~~~~~~~~~~~~~~~~~~~~~
%~~~~~~~~~~~~~~~~~~~~~~~~~~~~~~~~~~~~~~~~~~~~~~~~~~~~~~~~~~~~~~~~~~~~~~
%~~~~~~~~~~~~~~~~~~~~~~~~~~~~~~~~~~~~~~~~~~~~~~~~~~~~~~~~~~~~~~~~~~~~~~
%~~~~~~~~~~~~~~~~~~~~~~~~~~~~~~~~~~~~~~~~~~~~~~~~~~~~~~~~~~~~~~~~~~~~~~
%~~~~~~~~~~~~~~~~~~~~~~~~~~~~~~~~~~~~~~~~~~~~~~~~~~~~~~~~~~~~~~~~~~~~~~
\part{Engineering Mechanics-Dynamics: Kinetics of particles}
% Inertia of bodies
	% Part3_c01_inertia
\chapter{Inertia}

\begin{flushright}
\begin{minipage}[h]{10.5cm}
% \raggedright
{\small

\textbf{Generic understanding:} Inertia is that tendency of anything to maintain a constancy in its current state. \textit{In other words}, inertia is that tendency of anything to resist a change in the constancy of its current state. \textbf{Inertia of a body - Rectilinear/Curvilinear motion - Perspective 1:} \textit{The inertia of a body is that tendency of the body to a maintain constancy in its current state of uniform motion}. \textbf{The inertia of a body - Rectilinear motion - Perspective 2a:}  \textit{The inertia of a body is that tendency of the body to maintain its current state of rectilinear motion, that is, to either maintain its state of rest or uniform rectilinear motion \footnote{Here, uniform motion means motion with constant velocity, hence implying zero-acceleration either in/opposite to the direction of motion.}} \textbf{The inertia of a body - Curvilinear motion - Perspective 2b:}  \textit{The inertia of a body is that tendency of the body to maintain its current state of curvilinear motion, that is, to either maintain its state of rest or of uniform curvilinear motion \footnote{Here, uniform curvilinear motion means curvilinear motion with constant tangential velocity, hence implying zero-tangential-acceleration (in the case of curvilinear motion) either in/opposite to the direction of motion.}} \textbf{The inertia of a body - Perspective 3:}  Combining the above perspectives 2a and 2b, and defining the motion of the motion of the body of finite dimensions\footnote{Here motion of the body is defined by its center of gravity, G} by velocity vector $\vec{\mathbf{v}}$ \textit{The inertia is that tendency of the body to maintain its current state of $\vec{\mathbf{v}}$, that is to resist any change in $\vec{\mathbf{v}}$ as the time progresses}.

}
\end{minipage}

\end{flushright}

\textbf{Consider rectilinear motion of a body momentarily at rest}. When a force acts on it \footnote{Here, it is assumed that this applied force is the vectorial summation of all the forces applied on the body and also that such a resultant passes through the center of gravity (G) of the body; in order to simplify the analysis by removing any rotational components of motion that would otherwise manifest had the resultant were to not pass through G} \footnote{The resultant represents the resultant of the resultants of all applied and reactive (example, drag force, frictional resistance force, etc) as well}, it tends to move \footnote{The characteristics of this motion is defined by the motion parameters such as velocity and acceleration}, but however, the body does offer a resistance to the motion being imparted on it. The amount of resistance offered by the body to change its state of rest is decided by the mass content of the body and by what value of acceleration \footnote{Since the body is trying to move from rest, acceleration is positive} that the applied force is trying to impart to the body. As a consequence, the corresponding acceleration which the body acquires is decided by the mass content of the body itself and by the amount of the applied force as well. 

\begin{flushright}
\begin{minipage}[h]{10.5cm}
\small{
\textbf{Analysis 1a and 1b}: Had the mass of the object been any lesser, then for the same applied force, it would have acquired a greater value of acceleration in the direction of the applied force. On the other hand, had the mass of the body been any greater, then for the same applied force, it would have acquired a lesser value of acceleration in the direction of the applied force.

\textbf{Analysis 2}: When two bodies A and B of equal masses $m_A$ and $m_B$, such that $m_A = m_B$, then for the same force, there will be same values of accelerations.

\textbf{Analysis 3 - Explanations}: When two bodies A and B of equal masses are acted by the inequal forces, then bodies A and B acquire inequal value of acceleration. \textbf{Analysis 3 - Mathematical treatment}: When two bodies A and B of equal masses $m_A$ and $m_B$, such that $m_A = m_B$, are acted upon by the forces $\vec{\mathbf{F}_A}$ and $\vec{\mathbf{F}_B}$ ($|\vec{\mathbf{F}_A}|>|\vec{\mathbf{F}_B}|$), then $|\vec{\mathbf{a}_A}|>|\vec{\mathbf{a}_B}|$, that is the body A acquires greater value of acceleration than B. \textbf{Analysis 3 - Observation}: \textcolor{blue}{In other words, to accelerate a mass body to greater and greater values, higher and higher forces are needed}. \textbf{Analysis 3 - Conclusion}: \textcolor{red}{This dependency between force needed and acceleration to be acquired/imparted is found to be directly proportional and hence linear, and the constant of proportionality of this relationship is nothing but the mass of the object itself}. \textbf{Analysis 3 - Graphical interpretation} IMAGE. Hence, \textcolor{cyan}{the mass of the object can be seen as the slope of the above linear relationship between $\vec{\mathbf{F}}$ and $\vec{\mathbf{a}}$.}
}
\end{minipage}
\end{flushright}
TEXT HERE 
\begin{flushright}
\begin{minipage}[h]{10.5cm}
% \raggedright
{\small

\textbf{Analysis 1 - Explanations}: But why bring in acceleration here any way? \textit{We saw in the definition of inertia of a body that it not only represents the resistance offered by the body against the forces which are trying to change its state of rest, but also represents the resistance offered against the forces which are trying to change its state of motion under uniform velocity}. When a body in rectilinear motion is getting displaced by equal amounts in any arbitrarily chosen different but equal intervals of time, and if any unbalanced force now starts acting on this body and stays acting on the body (After the completion of the final time interval), then after this instant of time marking the application of this force, for any two arbitrarily chosen different but new equal intervals of time, the body undergoes in-equal displacements and hence the velocity of such body is different at different instances of time, both within the time intervals for which the applied force exists on the body. \textbf{Analysis 1 - Mathematical treatment - A}: If the displacements in the time interval $t_{12}$ is $\vec{\mathbf{s}}_{12}$  and that in $t_{23}$ is $\vec{\mathbf{s}}_{23}$, and if the time intervals and hence the corresponding displacements were to tend to zero under the limit, then the instantaneous values of velocity are $\vec{\mathbf{v}}_{12}$ = $d\vec{\mathbf{s}}_{12}/dt_{12}$, where $dt_{12}$ is an infinitesimal value of $t_{12}$ and $d\vec{\mathbf{s}}_{12}$ is the corresponding infinitesimal value of $\vec{\mathbf{s}}_{12}$; and $\vec{\mathbf{v}}_{23}$ = $d\vec{\mathbf{s}}_{23}/dt_{23}$, where $dt_{23}$ is the infinitesimal value of $t_{23}$ and $d\vec{\mathbf{s}}_{23}$ is the corresponding infinitesimal value of $\vec{\mathbf{s}}_{23}$. Here, $dt_{12}=dt_{23}$. After some finite instant of time, an unbalanced constant force starts acting on the body and continues to act on the same body. Now, let the intervals under observation be AB and BC on the same trajectory of the particle. It is to be noted that the particle reaches point A after passing through point 2 on the trajectory. \textbf{Analysis 1 - Mathematical treatment - B}: If the displacements in the time interval $t_{AB}$ is $\vec{\mathbf{s}}_{AB}$  and that in $t_{BC}$ is $\vec{\mathbf{s}}_{BC}$, then the instantaneous values of velocity are $\vec{\mathbf{v}}_{AB}$ = $d\vec{\mathbf{s}}_{AB}/dt_{AB}$, where $dt_{AB}$ is an infinitesimal value of $t_{AB}$ and $d\vec{\mathbf{s}}_{AB}$ is the corresponding infinitesimal value of $\vec{\mathbf{s}}_{AB}$; and $\vec{\mathbf{v}}_{BC}$ = $d\vec{\mathbf{s}}_{BC}/dt_{BC}$, where $dt_{BC}$ is the infinitesimal value of $t_{BC}$ and $d\vec{\mathbf{s}}_{BC}$ is the corresponding infinitesimal value of $\vec{\mathbf{s}}_{BC}$. Since clearly, $\vec{\mathbf{s}}_{AB}$ and $\vec{\mathbf{s}}_{BC}$ are not same here, $\vec{\mathbf{v}}_{AB} \ne \vec{\mathbf{v}}_{BC}$ for $t_{AB} = t_{BC}$. Hence there exists a time rate of change of velocity which between A and C is $\vec{\mathbf{a}}_{AC} = d\vec{\mathbf{v}}_{AC}/dt_{AC} = d(\vec{\mathbf{s}}_{AC}/dt_{AC})/dt_{AC}$, where once again, $d\vec{\mathbf{v}}_{AC}$ is the change in velocity between AB and BC intervals and $dt_{AC}$ is the infinitesimal time interval taken for displacement between A and C.
}
\end{minipage}
\end{flushright}

\subsection{Invariance of mass with velocity: Direction and magnitude}
Write about Lorentz transformation here.

\section{Mass moments of inertia: $I_{xx}, I_{yy}, I_{zz}$}
So far we have explored the inertia as applicable to rectilinear motion. Now we will pay attention to the moment form of such inertia which are ever so important in motion involving any aspects of rotation.
\subsection{Conceptual development}
In a purely mechanistic perspective, a body is basically a state space, where the shape, density distribution, energy and Eulerian location form important variables. In the universe, this state space evolves in the positive temporal dimension. When there will be a cause, these variables evolve with time. Say we have a such a body having a finite mass distributed around a certain vector in Eulerian space, then fore the cause to make this body execute rotation about the axis, 
\subsection{Mathematical development}
\subsection{Physical interpretations}
\subsection{Advanced treatise 1: Working with variable mass}
\subsection{Advanced treatise 2: Working with flexible bodies}
\subsection{Advanced treatise 3: Working with flexible bodies with variable mass}


\section{Products of mass moments of inertia: $I_{xy}, I_{yz}, I_{zx}$}
\subsection{Conceptual development}
\subsection{Mathematical development}
\subsection{Physical interpretations}

\section{Inertia tensor}
\subsection{Conceptual development}
\subsection{Mathematical development}
\subsection{Physical interpretations}

\section{Principal moments and axes of inertia}
\subsection{Conceptual development}
\subsection{Mathematical development}
\subsection{Physical interpretations}
%~~~~~~~~~~~~~~~~~~~~~~~~~~~~~~~~~~~~~~~~~~~~~~~~~~~~~~~~~~~~~~~~~~~~~~
%~~~~~~~~~~~~~~~~~~~~~~~~~~~~~~~~~~~~~~~~~~~~~~~~~~~~~~~~~~~~~~~~~~~~~~
%~~~~~~~~~~~~~~~~~~~~~~~~~~~~~~~~~~~~~~~~~~~~~~~~~~~~~~~~~~~~~~~~~~~~~~
%~~~~~~~~~~~~~~~~~~~~~~~~~~~~~~~~~~~~~~~~~~~~~~~~~~~~~~~~~~~~~~~~~~~~~~
%~~~~~~~~~~~~~~~~~~~~~~~~~~~~~~~~~~~~~~~~~~~~~~~~~~~~~~~~~~~~~~~~~~~~~~
%~~~~~~~~~~~~~~~~~~~~~~~~~~~~~~~~~~~~~~~~~~~~~~~~~~~~~~~~~~~~~~~~~~~~~~
%~~~~~~~~~~~~~~~~~~~~~~~~~~~~~~~~~~~~~~~~~~~~~~~~~~~~~~~~~~~~~~~~~~~~~~
%~~~~~~~~~~~~~~~~~~~~~~~~~~~~~~~~~~~~~~~~~~~~~~~~~~~~~~~~~~~~~~~~~~~~~~
%~~~~~~~~~~~~~~~~~~~~~~~~~~~~~~~~~~~~~~~~~~~~~~~~~~~~~~~~~~~~~~~~~~~~~~
%~~~~~~~~~~~~~~~~~~~~~~~~~~~~~~~~~~~~~~~~~~~~~~~~~~~~~~~~~~~~~~~~~~~~~~
%~~~~~~~~~~~~~~~~~~~~~~~~~~~~~~~~~~~~~~~~~~~~~~~~~~~~~~~~~~~~~~~~~~~~~~
%~~~~~~~~~~~~~~~~~~~~~~~~~~~~~~~~~~~~~~~~~~~~~~~~~~~~~~~~~~~~~~~~~~~~~~
\part{Engineering Mechanics-Dynamics: Kinematics of rigid bodies}
%~~~~~~~~~~~~~~~~~~~~~~~~~~~~~~~~~~~~~~~~~~~~~~~~~~~~~~~~~~~~~~~~~~~~~~
%~~~~~~~~~~~~~~~~~~~~~~~~~~~~~~~~~~~~~~~~~~~~~~~~~~~~~~~~~~~~~~~~~~~~~~
%~~~~~~~~~~~~~~~~~~~~~~~~~~~~~~~~~~~~~~~~~~~~~~~~~~~~~~~~~~~~~~~~~~~~~~
%~~~~~~~~~~~~~~~~~~~~~~~~~~~~~~~~~~~~~~~~~~~~~~~~~~~~~~~~~~~~~~~~~~~~~~
%~~~~~~~~~~~~~~~~~~~~~~~~~~~~~~~~~~~~~~~~~~~~~~~~~~~~~~~~~~~~~~~~~~~~~~
%~~~~~~~~~~~~~~~~~~~~~~~~~~~~~~~~~~~~~~~~~~~~~~~~~~~~~~~~~~~~~~~~~~~~~~
%~~~~~~~~~~~~~~~~~~~~~~~~~~~~~~~~~~~~~~~~~~~~~~~~~~~~~~~~~~~~~~~~~~~~~~
%~~~~~~~~~~~~~~~~~~~~~~~~~~~~~~~~~~~~~~~~~~~~~~~~~~~~~~~~~~~~~~~~~~~~~~
%~~~~~~~~~~~~~~~~~~~~~~~~~~~~~~~~~~~~~~~~~~~~~~~~~~~~~~~~~~~~~~~~~~~~~~
%~~~~~~~~~~~~~~~~~~~~~~~~~~~~~~~~~~~~~~~~~~~~~~~~~~~~~~~~~~~~~~~~~~~~~~
%~~~~~~~~~~~~~~~~~~~~~~~~~~~~~~~~~~~~~~~~~~~~~~~~~~~~~~~~~~~~~~~~~~~~~~
%~~~~~~~~~~~~~~~~~~~~~~~~~~~~~~~~~~~~~~~~~~~~~~~~~~~~~~~~~~~~~~~~~~~~~~
\part{Engineering Mechanics-Dynamics: Kinetics of rigid bodies}
%~~~~~~~~~~~~~~~~~~~~~~~~~~~~~~~~~~~~~~~~~~~~~~~~~~~~~~~~~~~~~~~~~~~~~~
%~~~~~~~~~~~~~~~~~~~~~~~~~~~~~~~~~~~~~~~~~~~~~~~~~~~~~~~~~~~~~~~~~~~~~~
%~~~~~~~~~~~~~~~~~~~~~~~~~~~~~~~~~~~~~~~~~~~~~~~~~~~~~~~~~~~~~~~~~~~~~~
%~~~~~~~~~~~~~~~~~~~~~~~~~~~~~~~~~~~~~~~~~~~~~~~~~~~~~~~~~~~~~~~~~~~~~~
%~~~~~~~~~~~~~~~~~~~~~~~~~~~~~~~~~~~~~~~~~~~~~~~~~~~~~~~~~~~~~~~~~~~~~~
%~~~~~~~~~~~~~~~~~~~~~~~~~~~~~~~~~~~~~~~~~~~~~~~~~~~~~~~~~~~~~~~~~~~~~~
%~~~~~~~~~~~~~~~~~~~~~~~~~~~~~~~~~~~~~~~~~~~~~~~~~~~~~~~~~~~~~~~~~~~~~~
%~~~~~~~~~~~~~~~~~~~~~~~~~~~~~~~~~~~~~~~~~~~~~~~~~~~~~~~~~~~~~~~~~~~~~~
%~~~~~~~~~~~~~~~~~~~~~~~~~~~~~~~~~~~~~~~~~~~~~~~~~~~~~~~~~~~~~~~~~~~~~~
%~~~~~~~~~~~~~~~~~~~~~~~~~~~~~~~~~~~~~~~~~~~~~~~~~~~~~~~~~~~~~~~~~~~~~~
%~~~~~~~~~~~~~~~~~~~~~~~~~~~~~~~~~~~~~~~~~~~~~~~~~~~~~~~~~~~~~~~~~~~~~~
%~~~~~~~~~~~~~~~~~~~~~~~~~~~~~~~~~~~~~~~~~~~~~~~~~~~~~~~~~~~~~~~~~~~~~~
\part{Analytical Mechanics}
% Mathematical preliminaries
% Co-ordinates
% Lagrangian mechanics
% Hamiltonian mechanics
% Routhian mechanics
%~~~~~~~~~~~~~~~~~~~~~~~~~~~~~~~~~~~~~~~~~~~~~~~~~~~~~~~~~~~~~~~~~~~~~~
%~~~~~~~~~~~~~~~~~~~~~~~~~~~~~~~~~~~~~~~~~~~~~~~~~~~~~~~~~~~~~~~~~~~~~~
%~~~~~~~~~~~~~~~~~~~~~~~~~~~~~~~~~~~~~~~~~~~~~~~~~~~~~~~~~~~~~~~~~~~~~~
%~~~~~~~~~~~~~~~~~~~~~~~~~~~~~~~~~~~~~~~~~~~~~~~~~~~~~~~~~~~~~~~~~~~~~~
%~~~~~~~~~~~~~~~~~~~~~~~~~~~~~~~~~~~~~~~~~~~~~~~~~~~~~~~~~~~~~~~~~~~~~~
%~~~~~~~~~~~~~~~~~~~~~~~~~~~~~~~~~~~~~~~~~~~~~~~~~~~~~~~~~~~~~~~~~~~~~~
%~~~~~~~~~~~~~~~~~~~~~~~~~~~~~~~~~~~~~~~~~~~~~~~~~~~~~~~~~~~~~~~~~~~~~~
%~~~~~~~~~~~~~~~~~~~~~~~~~~~~~~~~~~~~~~~~~~~~~~~~~~~~~~~~~~~~~~~~~~~~~~
%~~~~~~~~~~~~~~~~~~~~~~~~~~~~~~~~~~~~~~~~~~~~~~~~~~~~~~~~~~~~~~~~~~~~~~
%~~~~~~~~~~~~~~~~~~~~~~~~~~~~~~~~~~~~~~~~~~~~~~~~~~~~~~~~~~~~~~~~~~~~~~
%~~~~~~~~~~~~~~~~~~~~~~~~~~~~~~~~~~~~~~~~~~~~~~~~~~~~~~~~~~~~~~~~~~~~~~
%~~~~~~~~~~~~~~~~~~~~~~~~~~~~~~~~~~~~~~~~~~~~~~~~~~~~~~~~~~~~~~~~~~~~~~
\part{Engineering Mechanics-Dynamics: Vibration of rigid bodies}
\part{Engineering Mechanics-Dynamics: Flexible body kinematics}
\part{Engineering Mechanics-Dynamics: Flexible body kinetics}
\part{Engineering Mechanics-Dynamics: Vibration of flexible bodies}
%~~~~~~~~~~~~~~~~~~~~~~~~~~~~~~~~~~~~~~~~~~~~~~~~~~~~~~~~~~~~~~~~~~~~~~
%~~~~~~~~~~~~~~~~~~~~~~~~~~~~~~~~~~~~~~~~~~~~~~~~~~~~~~~~~~~~~~~~~~~~~~
%~~~~~~~~~~~~~~~~~~~~~~~~~~~~~~~~~~~~~~~~~~~~~~~~~~~~~~~~~~~~~~~~~~~~~~
%~~~~~~~~~~~~~~~~~~~~~~~~~~~~~~~~~~~~~~~~~~~~~~~~~~~~~~~~~~~~~~~~~~~~~~
%~~~~~~~~~~~~~~~~~~~~~~~~~~~~~~~~~~~~~~~~~~~~~~~~~~~~~~~~~~~~~~~~~~~~~~
%~~~~~~~~~~~~~~~~~~~~~~~~~~~~~~~~~~~~~~~~~~~~~~~~~~~~~~~~~~~~~~~~~~~~~~
%~~~~~~~~~~~~~~~~~~~~~~~~~~~~~~~~~~~~~~~~~~~~~~~~~~~~~~~~~~~~~~~~~~~~~~
%~~~~~~~~~~~~~~~~~~~~~~~~~~~~~~~~~~~~~~~~~~~~~~~~~~~~~~~~~~~~~~~~~~~~~~
%~~~~~~~~~~~~~~~~~~~~~~~~~~~~~~~~~~~~~~~~~~~~~~~~~~~~~~~~~~~~~~~~~~~~~~
%~~~~~~~~~~~~~~~~~~~~~~~~~~~~~~~~~~~~~~~~~~~~~~~~~~~~~~~~~~~~~~~~~~~~~~
%~~~~~~~~~~~~~~~~~~~~~~~~~~~~~~~~~~~~~~~~~~~~~~~~~~~~~~~~~~~~~~~~~~~~~~
%~~~~~~~~~~~~~~~~~~~~~~~~~~~~~~~~~~~~~~~~~~~~~~~~~~~~~~~~~~~~~~~~~~~~~~
\appendix
\include{a01}
\backmatter
\include{b01}
\printindex
\end{document}