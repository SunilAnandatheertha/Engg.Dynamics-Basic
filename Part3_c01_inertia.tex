% Part3_c01_inertia
\chapter{Inertia}

\begin{flushright}
\begin{minipage}[h]{10.5cm}
% \raggedright
{\small

\textbf{Generic understanding:} Inertia is that tendency of anything to maintain a constancy in its current state. \textit{In other words}, inertia is that tendency of anything to resist a change in the constancy of its current state. \textbf{Inertia of a body - Rectilinear/Curvilinear motion - Perspective 1:} \textit{The inertia of a body is that tendency of the body to a maintain constancy in its current state of uniform motion}. \textbf{The inertia of a body - Rectilinear motion - Perspective 2a:}  \textit{The inertia of a body is that tendency of the body to maintain its current state of rectilinear motion, that is, to either maintain its state of rest or uniform rectilinear motion \footnote{Here, uniform motion means motion with constant velocity, hence implying zero-acceleration either in/opposite to the direction of motion.}} \textbf{The inertia of a body - Curvilinear motion - Perspective 2b:}  \textit{The inertia of a body is that tendency of the body to maintain its current state of curvilinear motion, that is, to either maintain its state of rest or of uniform curvilinear motion \footnote{Here, uniform curvilinear motion means curvilinear motion with constant tangential velocity, hence implying zero-tangential-acceleration (in the case of curvilinear motion) either in/opposite to the direction of motion.}} \textbf{The inertia of a body - Perspective 3:}  Combining the above perspectives 2a and 2b, and defining the motion of the motion of the body of finite dimensions\footnote{Here motion of the body is defined by its center of gravity, G} by velocity vector $\vec{\mathbf{v}}$ \textit{The inertia is that tendency of the body to maintain its current state of $\vec{\mathbf{v}}$, that is to resist any change in $\vec{\mathbf{v}}$ as the time progresses}.

}
\end{minipage}

\end{flushright}

\textbf{Consider rectilinear motion of a body momentarily at rest}. When a force acts on it \footnote{Here, it is assumed that this applied force is the vectorial summation of all the forces applied on the body and also that such a resultant passes through the center of gravity (G) of the body; in order to simplify the analysis by removing any rotational components of motion that would otherwise manifest had the resultant were to not pass through G} \footnote{The resultant represents the resultant of the resultants of all applied and reactive (example, drag force, frictional resistance force, etc) as well}, it tends to move \footnote{The characteristics of this motion is defined by the motion parameters such as velocity and acceleration}, but however, the body does offer a resistance to the motion being imparted on it. The amount of resistance offered by the body to change its state of rest is decided by the mass content of the body and by what value of acceleration \footnote{Since the body is trying to move from rest, acceleration is positive} that the applied force is trying to impart to the body. As a consequence, the corresponding acceleration which the body acquires is decided by the mass content of the body itself and by the amount of the applied force as well. 

\begin{flushright}
\begin{minipage}[h]{10.5cm}
\small{
\textbf{Analysis 1a and 1b}: Had the mass of the object been any lesser, then for the same applied force, it would have acquired a greater value of acceleration in the direction of the applied force. On the other hand, had the mass of the body been any greater, then for the same applied force, it would have acquired a lesser value of acceleration in the direction of the applied force.

\textbf{Analysis 2}: When two bodies A and B of equal masses $m_A$ and $m_B$, such that $m_A = m_B$, then for the same force, there will be same values of accelerations.

\textbf{Analysis 3 - Explanations}: When two bodies A and B of equal masses are acted by the inequal forces, then bodies A and B acquire inequal value of acceleration. \textbf{Analysis 3 - Mathematical treatment}: When two bodies A and B of equal masses $m_A$ and $m_B$, such that $m_A = m_B$, are acted upon by the forces $\vec{\mathbf{F}_A}$ and $\vec{\mathbf{F}_B}$ ($|\vec{\mathbf{F}_A}|>|\vec{\mathbf{F}_B}|$), then $|\vec{\mathbf{a}_A}|>|\vec{\mathbf{a}_B}|$, that is the body A acquires greater value of acceleration than B. \textbf{Analysis 3 - Observation}: \textcolor{blue}{In other words, to accelerate a mass body to greater and greater values, higher and higher forces are needed}. \textbf{Analysis 3 - Conclusion}: \textcolor{red}{This dependency between force needed and acceleration to be acquired/imparted is found to be directly proportional and hence linear, and the constant of proportionality of this relationship is nothing but the mass of the object itself}. \textbf{Analysis 3 - Graphical interpretation} IMAGE. Hence, \textcolor{cyan}{the mass of the object can be seen as the slope of the above linear relationship between $\vec{\mathbf{F}}$ and $\vec{\mathbf{a}}$.}
}
\end{minipage}
\end{flushright}
TEXT HERE 
\begin{flushright}
\begin{minipage}[h]{10.5cm}
% \raggedright
{\small

\textbf{Analysis 1 - Explanations}: But why bring in acceleration here any way? \textit{We saw in the definition of inertia of a body that it not only represents the resistance offered by the body against the forces which are trying to change its state of rest, but also represents the resistance offered against the forces which are trying to change its state of motion under uniform velocity}. When a body in rectilinear motion is getting displaced by equal amounts in any arbitrarily chosen different but equal intervals of time, and if any unbalanced force now starts acting on this body and stays acting on the body (After the completion of the final time interval), then after this instant of time marking the application of this force, for any two arbitrarily chosen different but new equal intervals of time, the body undergoes in-equal displacements and hence the velocity of such body is different at different instances of time, both within the time intervals for which the applied force exists on the body. \textbf{Analysis 1 - Mathematical treatment - A}: If the displacements in the time interval $t_{12}$ is $\vec{\mathbf{s}}_{12}$  and that in $t_{23}$ is $\vec{\mathbf{s}}_{23}$, and if the time intervals and hence the corresponding displacements were to tend to zero under the limit, then the instantaneous values of velocity are $\vec{\mathbf{v}}_{12}$ = $d\vec{\mathbf{s}}_{12}/dt_{12}$, where $dt_{12}$ is an infinitesimal value of $t_{12}$ and $d\vec{\mathbf{s}}_{12}$ is the corresponding infinitesimal value of $\vec{\mathbf{s}}_{12}$; and $\vec{\mathbf{v}}_{23}$ = $d\vec{\mathbf{s}}_{23}/dt_{23}$, where $dt_{23}$ is the infinitesimal value of $t_{23}$ and $d\vec{\mathbf{s}}_{23}$ is the corresponding infinitesimal value of $\vec{\mathbf{s}}_{23}$. Here, $dt_{12}=dt_{23}$. After some finite instant of time, an unbalanced constant force starts acting on the body and continues to act on the same body. Now, let the intervals under observation be AB and BC on the same trajectory of the particle. It is to be noted that the particle reaches point A after passing through point 2 on the trajectory. \textbf{Analysis 1 - Mathematical treatment - B}: If the displacements in the time interval $t_{AB}$ is $\vec{\mathbf{s}}_{AB}$  and that in $t_{BC}$ is $\vec{\mathbf{s}}_{BC}$, then the instantaneous values of velocity are $\vec{\mathbf{v}}_{AB}$ = $d\vec{\mathbf{s}}_{AB}/dt_{AB}$, where $dt_{AB}$ is an infinitesimal value of $t_{AB}$ and $d\vec{\mathbf{s}}_{AB}$ is the corresponding infinitesimal value of $\vec{\mathbf{s}}_{AB}$; and $\vec{\mathbf{v}}_{BC}$ = $d\vec{\mathbf{s}}_{BC}/dt_{BC}$, where $dt_{BC}$ is the infinitesimal value of $t_{BC}$ and $d\vec{\mathbf{s}}_{BC}$ is the corresponding infinitesimal value of $\vec{\mathbf{s}}_{BC}$. Since clearly, $\vec{\mathbf{s}}_{AB}$ and $\vec{\mathbf{s}}_{BC}$ are not same here, $\vec{\mathbf{v}}_{AB} \ne \vec{\mathbf{v}}_{BC}$ for $t_{AB} = t_{BC}$. Hence there exists a time rate of change of velocity which between A and C is $\vec{\mathbf{a}}_{AC} = d\vec{\mathbf{v}}_{AC}/dt_{AC} = d(\vec{\mathbf{s}}_{AC}/dt_{AC})/dt_{AC}$, where once again, $d\vec{\mathbf{v}}_{AC}$ is the change in velocity between AB and BC intervals and $dt_{AC}$ is the infinitesimal time interval taken for displacement between A and C.
}
\end{minipage}
\end{flushright}

\subsection{Invariance of mass with velocity: Direction and magnitude}
Write about Lorentz transformation here.

\section{Mass moments of inertia: $I_{xx}, I_{yy}, I_{zz}$}
So far we have explored the inertia as applicable to rectilinear motion. Now we will pay attention to the moment form of such inertia which are ever so important in motion involving any aspects of rotation.
\subsection{Conceptual development}
In a purely mechanistic perspective, a body is basically a state space, where the shape, density distribution, energy and Eulerian location form important variables. In the universe, this state space evolves in the positive temporal dimension. When there will be a cause, these variables evolve with time. Say we have a such a body having a finite mass distributed around a certain vector in Eulerian space, then fore the cause to make this body execute rotation about the axis, 
\subsection{Mathematical development}
\subsection{Physical interpretations}
\subsection{Advanced treatise 1: Working with variable mass}
\subsection{Advanced treatise 2: Working with flexible bodies}
\subsection{Advanced treatise 3: Working with flexible bodies with variable mass}


\section{Products of mass moments of inertia: $I_{xy}, I_{yz}, I_{zx}$}
\subsection{Conceptual development}
\subsection{Mathematical development}
\subsection{Physical interpretations}

\section{Inertia tensor}
\subsection{Conceptual development}
\subsection{Mathematical development}
\subsection{Physical interpretations}

\section{Principal moments and axes of inertia}
\subsection{Conceptual development}
\subsection{Mathematical development}
\subsection{Physical interpretations}