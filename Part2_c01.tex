\chapter{Introduction}

\subsection{Frame of reference}
A \textit{frame of reference} in $\mathbf{R}^3$ space is a reference system having three mutually independent co-ordinates which can uniquely identify any point in that space. A \textit{fixed frame of reference} $\mathbf{R_A}$ is one which is non-moving and fixed in space, does not translate or rotate, and also the one with respect to which or with reference to which, all other measurements of motion can be made. If there exists another frame of reference $\mathbf{R_B}$ who's orientation and/or motion (translation and/or rotation) is known with respect to the $R_A$, and if the motion of a point moving in space can be easily observed from $R_B$, then the observations made in $R_B$ has to be transformed using transformations defined by the orientation of $R_B$ with respect to $R_A$, in order for the observations from $R_B$ to be interpretable as observations from the fixed frame of reference $R_A$. When a frame of reference such as $R_B$ moves with constant zero (stationary) /non-zero velocity with respect to $R_A$, then $R_B$ is an inertial frame of reference. Thus, an \textit{inertial frame of reference} is one which has no relative motion with the fixed frame of reference or is one which has constant velocity measured from the fixed frame of reference. Other frames of references such as \textit{rotating frame of reference} are used in the kinematic and kinetic analysis, and they will be introduced, explained and used when necessary. This topic is discussed and explained in more details later in the text.
	
\subsection{Scalars and vectors}
A \textit{scalar} is a mathematical representation of any `quantity' (Physical/Non-Physical) having significance only in its magnitude, but not in direction. This does not mean there could be no direction. It only means no significance of direction is desired/attached. However, a \textit{vector} is a mathematical representation of a physical quantity which can only be completely specified when we have full knowledge of its magnitude (or mathematical behavior of its magnitude) and also its direction/sense (or mathematical behavior of its direction). Many entities associated with motion fall into the category of vectors. In contrast to a scalar, a specific co-ordinate system is needed to completely, mathematically and quantitatively specify a vector.
	
\subsection{Distance and displacement}
Consider an arbitrary point in space at a position P1(x1, y1, z1) making is way to another position P2(x2, y2, z2) in a certain path. To do so, this particle traveled a distance 's' along the path that it took to reach P2 from P1. Upon starting from P1 and reaching P2, the point is displaced along a direction defined by the spatial co-ordinates of P1 and P2 (and also on the position of the reference used), by a length which again defined by spatial co-ordinates of P1 and P2. This length is the Euclidean straight line distance between P1 and P2 taken in rectangular co-ordinate system and is defined as $|\mathbf{s}| = \sqrt[2]{(x_2-x_1)^2+(y_2-y_1)^2+(z_2-z_1)^2}$. If the reference used for observing the motion of point from P1 to P2 is the fixed frame of reference, then the 

\subsection{Speed and velocity} When particle moves in 

\subsection{Velocity}
Velocity  of an arbitrary point of interest is a measure of time rate of change of displacement of that point. 

\textit{Analysis - 1a}: For a particle, the knowledge of it's magnitude of velocity implies a  knowledge of the relation between the actual displacement underwent and the actual time taken. It means to have knowledge of \textit{how much the particle actually got displaced} in \textit{exactly that much amount of time}. We generalize this argument in analysis - 1b below.

\textit{Analysis - 1b}: For a particle, the knowledge of it's magnitude of velocity implies a  knowledge of the relation between any arbitrary displacement underwent and the corresponding time taken, provided that all measurements are done within the time interval throughout which the velicity remains constant. It means to have knowledge of \textit{how much the particle would get displaced} in \textit{exactly so much amount of time}.

If the measure of displacement is rectilinear, the velocity is rectilinear velocity $\vec{\mathbf{v}}$. If the measure of displacement is curvilinear, the velocity is tangential velocity $\vec{\mathbf{v}}_t$. However, if the measure of displacement is angular, the velocity is angular velocity $\vec{\mathbf{\omega}}$. The SI units of $\vec{\mathbf{v}}$ and $\vec{\mathbf{v}}_t$ is $m/s$ and that of $\vec{\mathbf{\omega}}$ is $rad/s$. Since, displacement is a vector, it follows from the rules of vector calculus that, velocity is also a vector.

\begin{flushright}
\begin{minipage}[h!]{10.5cm}
\small{
\textbf{Physical interpretation - 1}\\ \textit{The direction of velocity vector is the same as that of the direction of displacement vector}. A particle has started to execute rectilinear motion from point 0. It reaches point 1 and then to point 2. Let, time taken to reach points 1 and 2 be started to be recorded from point 0, and say they are $t_{01}$ and $t_{02}$; \textbf{O} be the origin of an inertial frame which is used to observe the motion of the particle as it moves from point 0 to point 2 (All our measurements and hence, the following statements are made with respect to \textbf{O}) and $\mathbf{r}_{1/O}$ and $\mathbf{r}_{2/O}$ be the position vectors of instantaneous positions of the particle at points 1 and 2. Clearly the difference $\Delta t_{02-01} = \Delta t_{2-1} = t_{02}-t_{01}$, ($+^{ve}$ scalar) and is the time elapsed as the particle moved from points 1 to 2. However, the difference $\Delta r_{(2/O)/(1/O)} = \Delta r_{2/1} = r_{2/O}-r_{1/O}$ is vectorial, who's magnitude is just the difference in magnitudes of the two vectors, but however, the direction of $\Delta r_{2/1}$ is that of the vector joining the 2nd position (point 2) from the 1st position (point 1). This direction is that of the direction of motion between points 1 and 2. Hence, the direction of velocity vector of the particle executing motion from point 1 to 2 is the same as that of the displacement vector between the two points. When the spatial and time intervals between the above two states, states 1 and 2, is shrink to infinitesimal values (by bringing point 2 infinitesimally close to point 1) to be represented by an infinitesimal position vector, $d\mathbf{r}_{2/1}$ and infinitesimal time interval $dt_{2-1}$, we get the instantaneous velocity vector at point 1, which can be represented by $\mathbf{v}_{1}$, given by the ratio of $d\mathbf{r}_{2/1}$ and $dt_{2-1}$. Hence,\[\mathbf{v}_{1}=\frac{d\mathbf{r}_{2/1}}{dt_{2-1}} \]
}
\end{minipage}
\end{flushright}

\textbf{Analysis - 1}:

\subsection{Constant velocity}

\subsubsection{Constant rectilinear velocity or Zero-rectilinear Acceleration}
Consider a particle executing rectilinear motion. When the velocity vector $\textbf{v}$ is constant, its time derivative, that is $\frac{d\mathbf{v}}{dt}$, equal to acceleration vector $\mathbf{a}$, is zero.

Please go through the `\textbf{Note on reading `analysis' sections}' provided in Preface section of this book before you go any further in reading the following analysis points.


\begin{flushright}
\begin{minipage}[h]{10.5cm}
\small{
\textbf{Analysis - 1}: If a particle moving from point A to point B, from time instance $t_A$ to $t_B$, that is, in a time interval of $t_{AB}=t_B-t_A$, displaces itself with respect to position A by a Euclidean straight line distance of $|\mathbf{s}_{AB}|$, and if its displacement from position A to any arbitrarily chosen intermediate point C on the path its takes from A to B is $|\mathbf{s}_{AC}|$ and the time taken for this displacement is $t_{AC}$, and if the ratio of $|\mathbf{s}_{AB|}$ to $t_{AB}$ equals the ratio of $|\mathbf{s}_{AC}|$ to $t_{AB}$, then the displacement underwent by such a particle between any two arbitrary positions (say P and Q) on the path under consideration, follows a linear relationship with the time taken by the particle to be displaced from P to Q and the rate at which this ratio changes with time is essentially zero since this ratio is a constant, and in such cases we say that the particle has a constant velocity and consequently has no or zero-acceleration. The linear relationship mentioned here can be expressed as: $|\mathbf{s}_{PQ}|=\left(\frac{|\mathbf{s}_{AB}|}{t_{AB}}\right)t_{PQ}=\left(\frac{|\mathbf{s}_{AC}|}{t_{AC}}\right)t_{PQ}$. The term in brackets is the slope of the linear relationship between $|\mathbf{s}_{PQ}|$ and $t_{PQ}$.

\textbf{Analysis - 2}: Given an arbitrary time interval $t_{AB}$ and another arbitrary time interval $t_{PQ}$, both belonging to the time elapsed between positions A and B on the path 1-2 and P and Q on the same path 1-2, then under constant velocity motion between 1 and 2, for $t_{AB}=t_{PQ}$, the displacement of the particle from A to B will equal that of P to Q.

\textbf{Analysis - 3}: If $|\mathbf{s}_{AB}|$ and $|\mathbf{s}_{PQ}|$ be any two arbitrarily chosen equal displacement magnitudes (between points A and B, and P and Q respectively) underwent by a particle moving along a certain space, then, the time taken by the particle to be displaced from A to B and that between P and Q, should be equal.
}
\end{minipage}
\end{flushright}






\subsection{Acceleration}


\subsection{Constant acceleration}: 
\subsection{Variable acceleration: a = a(t), a = a(v) and a = a(s)}
\subsection{Practical cases of constant acceleration}
\subsection{Practical cases of variable acceleration}
\subsection{Working with variable acceleration problems}
%~~~~~~~~~~~~~~~~~~~~~~~~~~~~~~~~~~~~~~~~~~~~~~~~~~~~~~~~~~
%~~~~~~~~~~~~~~~~~~~~~~~~~~~~~~~~~~~~~~~~~~~~~~~~~~~~~~~~~~
\section{Co-ordinate systems}
\section{Rectangular coordinates}
\section{Normal and tangential co-ordinates}
\section{Polar co-ordinates}
\section{Space curvilinear motion}
\section{Relative motion with translating axis}
\section{Constrained motion of connected bodies}